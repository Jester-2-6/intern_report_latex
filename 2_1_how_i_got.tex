%!TEX root = ./intern_report.tex

\subsection{How I got the Opportunity}

\paragraph{}
A long time before the industrial training selection, I had heard Paraqum Technologies was one of the best places to learn about the electronics industry in Sri Lanka and Its CEO, Dr. Ajith Pasqual had taught a module in the university that sparked an interest in me about the subject of silicon design. Therefore, when Paraqum Technologies was listed as an open CV company, I did not hesitate to submit a CV, which got selected by the company staff who then interviewed me thoroughly in their office and sometime later, informed me that I had been selected to the Wave Computing Division.

\paragraph{}
At the start of the Internship, I was placed under the supervision of Eng. Achintha Ihalage, an application engineer whose original task was to handle the timing and constraints of the DPU chip. He walked me through the basics of setting up the Wave Computing workspace, company work ethics and other needed technical skills. He also introduced us to the DPU hardware and other proprietary technologies by Wave. 

\paragraph{}
During the second week, Eng. Henrik Esbenson, who is in charge of the Sri Lankan team at Wave HQ visited the office and demonstrated his ideas for projects. One of these was the Python - Wave Flow Graph translator, also known as Py2WFG. I volunteered to take that project and Eng. Henrik allocated me the necessary resources of the company including support teams and software tools to carry on the project.

\paragraph{}
This project soon became popular among the crowd of wave computing and I developed it according to the requirements and feedback. The amount of work was rather large but I managed to complete the project and hand it over by the time Internship ended. This project will most likely be adopted by full time developers and expanded to probably replace the existing design flow.

